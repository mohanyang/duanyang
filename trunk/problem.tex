\ifx \allfiles \undefined
\documentclass{article}
\usepackage{booktabs}
\usepackage{multirow}
\usepackage{graphicx}

\begin{document}
\title{Problem Formulation}
\maketitle \else \fi

\section{Problem Formulation}\label{sec:problem}
We formulate the social network in twitter as a direct graph $G(V,E)$,
a node $u$ represents a user in twitter and a direct edge $(u,v)$ indicates
that user $u$ is following user $v$.

For each individual user $u$, we collected the user's tweets
$T(u)=\{t_1, t_2, \cdots\}$.
Each tweet contains a set of words in $W=\{w_1, w_2, \cdots\}$.
We use a vector $\phi_u$ represents the word feature generated by the user.
Here $\phi_{u,i}=1$ indicates that word $w_i$ appeared in $u$'s tweets,
and $\phi_{u,i}=0$ indicates that word $w_i$ did not appeared in $u$'s tweets.
To simplify our problem, we ignore the number of occurrences of each word $w_i$.

In addition to a user's short bio,
we also collected a series of characteristic keywords
$z_1, z_2, \cdots$ from his short bio. For example, a student study computer
science in UCLA may write ``Computer science student at UCLA'' in his short bio.
We extract his characteristic profile as
$z_1=$Computer Science, $z_2$=student, $z_3=$UCLA.
Because computer could not generate these characteristic profile automatically
and accurately, we choose some phrases manually and consider other words in
profile as single keyword for the user.

Given a keyword or phrase $z_i$, we call users which write $z_i$ in their
short bio directly as nodes $L={l_1, l_2, \cdots, l_n}$,
while we call other users without the $p_i$ as nodes $D={d_1, d_2, \cdots, d_m}$.
The total user set we have is $C=D \cup L$.
Although some users in nodes set $D$ have not write $p_i$ in their profile,
he actually owns the keyword or phrase in real life.
As the example before, a student in UCLA could also write
``loving being a BRUIN and loving God!'' in her short bio.

Now our task is to design some algorithms that for a specify keyword or
phrase $z_i$, predict how user $u$ in nodes set $D$ is related with $z_i$.
In order to achieve this goal, we may assign some weight on words appeared
in user's tweets, short bio, or locations. Additionally, we may analysis
user's link relationship with others.
Our result would be a sorted list that ranks the user with high probability
having $z_i$ in real life at first.

\ifx \allfiles \undefined
\end{document}
\fi

\ifx \allfiles \undefined      
\documentclass{article}
\usepackage{booktabs}
\usepackage{multirow}
\usepackage{graphicx}
\usepackage{subfigure}

\begin{document}
\title{Conclusion}
\maketitle \else \fi

\section{Conclusion}\label{sec:conclusion}
Effectively estimating user profile and accordingly recommending service or suggesting friends are fundamental to all social networks. In this report, we have shown that the user's short bio is highly related to user's friendship and user's tweets. We presented three simple ranking approaches and a co-training framework that leverage both friendship and tweets evidence to solve the task purposed in our report. The graph approach analysis user profile from his followings and followers since that similar users are more likely to connected with each other. The Bayes approach extract the semantics of individual message that allow for the generation of user profile information of a given concept. Given the co-training framework, it is easy for us to combine two different approaches and obtain a better ranking result with limited positive training examples. The experiments results on twitter social network demonstrate that simple algorithms perform very well for our task. Additionally, we learned the pros and cons of different approaches from the discussion.

The co-training framework that combines the knowledge from graph structure and tweets information is not limited to predict user profiles. It can be applied to many other problems that require learning to rank nodes in a graph. There are some interest future research directions: First, the users are equally important in our model based on graph structure. However, we find many inactive users when we label the users in the ranked list. In the future, we may assume users have different weight during the training process. Thus, there may exists the underlying mechanism of how the interactions and information between users related to their personal profile. Second, it is interesting to apply our algorithms in friends recommendation system. Currently, most friends recommendation systems were based on number of users' mutual friends. The co-training approach could leverage users' mutual friends information and other user behaviors such as tweets, profiles. I think it is very helpful to build such framework for friends recommendation.

\ifx \allfiles \undefined
\end{document}
\fi

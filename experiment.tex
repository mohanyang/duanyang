\section{Experiments}\label{sec:experiment}


%I think we need rewrite this part. Thanks.

%Our crawler will begin with Twitter users from four universities including UCLA, USC, Stanford and MIT, and perform a two-step random walk starting from these users to obtain a larger set of different background users. Every user's profile, followers and followings information will be retrieved. The crawler will also retrieve these users' tweets posted after a specific time.
%We will analysis the number of followers, followings, tweets and retweets for each user, and quantify a user's influence on these metrics using a PageRank like mechanism.
%We will analysis the tweet keywords for each user, and provide a metric to evaluate the tweets similarity between two users. We will also compare the interest difference between users in different universities.
%We will quantify two users' closeness using their tweet and reply (regarding retweet as a kind of reply) behavior data. Our model is similar to the model used in predicting a chess game result based on two players' past campaign results \cite{elo1986rating}.
%We will propose an algorithm for extracting a user's hidden profile. Each edge will be weighted by the closeness and tweets similarity of the corresponding two users. The algorithm performs a random walk on the weighted graph to collect all possible hidden profile for a specific user.

\subsection{Data Collection}
We selected 20 seed users from each of the four universities, including UCLA, USC, Stanford and MIT. User profile (id, location, screen name, number of followings and followers, and a short biography) of these seed users was crawled using the Twitter API. Starting from these seed users, a two-step random walk is performed to retrieve their followers and their followers' followers. If a new user from these four universities is discovered during the procedure, this user is marked as a seed user and another two-step random walk starting from this user is performed. The crawler has collected more than 540,000 users and 780,000,000 following relations. It has also collected the most recent 20 tweets for each user.

\subsection{Data Analysis}
Keywords are extracted from user profiles. Keyword frequencies are calculated, and some keywords are manually labeled into three categories, namely location (CA, Los Angeles, NY, etc.), occupation (CEO, writer, blogger, student, etc.) and affiliation (UCLA, USC, Stanford, MIT, Google, etc.). Following is some statistics about the users in different universities (a subset of affiliation).

\subsubsection{Who follows whom on twitter?}
Table 1 shows the average numbers of followings in different universities with respect to users in different universities. 'Other' represents that the user does not mention any of these four universities in his biography. For example, the third row and second column means 38.17 followings of a UCLA user is in other universities on average. It is shown that the number of followers in the same university is ten times larger than that in different universities. However, this only counts for 5\% among all a user's followers. A possible reason might be that many users do not write their universities in their biographies.

%Maybe we don't need this

%\subsubsection{Who replies/retweets to whom on twitter?}
%More than 75,000 replies/retweets are collected. 86\% of them are between a user in these four universities and a user in affiliation "Other". For the rest replies/retweets, 98\% of them are between users in the same universities.
